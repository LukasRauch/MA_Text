\documentclass[german,a4paper,12pt,oneside]{scrbook}
\usepackage[latin1]{inputenc}
\usepackage{amsmath, amsthm, amssymb}
\usepackage[ngerman]{babel}
\usepackage{marvosym}
\usepackage{graphics}
\usepackage{natbib}

\newtheorem{satz}{Satz}[chapter]
\theoremstyle{definition} 
\newtheorem{definition}[satz]{Definition} 
\theoremstyle{definition} 
\newtheorem{lemma}[satz]{Lemma} 
\theoremstyle{definition} 
\newtheorem{bemerkung}[satz]{Bemerkung}
\theoremstyle{definition} 
\newtheorem{korollar}[satz]{Korollar} 
\theoremstyle{definition}
\newtheorem{beispiel}[satz]{Beispiel} 
\theoremstyle{definition} 
\newtheorem{algorithmus}{Algorithmus} 
\newenvironment{beweis}{\begin{proof}[Beweis]}{\end{proof}}


\begin{document}

% Titelseite
% \pagestyle{empty}       % keine Seitennummer
%   \parbox{1.5cm}{\resizebox*{110pt}{!}{\includegraphics{TUM_Logo.pdf}}}\hspace{310pt}%
%   \parbox{1.5cm}{\resizebox*{90pt}{!}{\includegraphics{Institutslogo.pdf}}}%
\vspace*{1.5cm}
\begin{center}
{\Huge Technische Universit\"at M\"unchen} 
\\
\vspace*{1.5cm}
{\huge \sc{ Fakult\"at f\"ur Mathematik}} 
\\
\vspace*{3cm}
{\Huge {\bf [Titel der Masterarbeit]}}
\\
\vspace*{3cm}
{\Large Masterarbeit}\linebreak \\ 
{\Large von}\linebreak \\
{\Large [Lukas Rauch]}\\
\vspace*{3cm}
{\Large 
\begin{tabular}{ll}
Aufgabensteller: & Prof. Dr. [Vorname Nachname]\\
Betreuer: & [Vorname Nachname]\\
Abgabetermin: & [Tag. Monat. Jahr]
\end{tabular}
}
\end{center}
\newpage    % Seitenwechsel

% Seite 2
\vspace*{18cm}
\noindent
Hiermit erkl\"are ich, dass ich die Masterarbeit selbstst\"andig und nur mit den angegebenen Hilfsmitteln angefertigt habe.
\\
\\
\\
\\
\\
München, den [Datum]
\newpage

% vertikaler Leerraum
\vspace*{2.2cm}
\noindent %kein Einzug
{\Huge {\bf Danksagung}} \\
\vspace*{1.6cm} \\
% Seitennummerierung "r�misch
\pagenumbering{roman}
% Kopfzeilen (automatisch erzeugt)
\pagestyle{headings}
[Text der Danksagung]

% Seite 3
\newpage
\section*{Zusammenfassung auf Deutsch}
[Text der Zusammenfassung]
\section*{Zusammenfassung auf Englisch}
[Summary of the thesis]
%Seite 4
\newpage
\tableofcontents  


\pagenumbering{arabic}  % Nummerierung der Seiten in 'arabisch' % neues Kapitel mit Namen "Introduction"
\chapter{Einleitung}  \setcounter{page}{1}   % setzt Seitenzaehlung auf 1
% Zitate aus B\"uchern werden so gemacht, siehe in \cite{schoenbucher03} oder \cite{marshall67}.

\chapter{Herleitung eines Truss Elements}

im Folgendne wird stellvertretend für die Herleitung eines IGA Elements die Herleitung eines Truss-Elements 
dargestellt. 

Die Vorgehensweise ist vergleichbar mit der Herleitung des Nichtlinearen Beam-Elements in diesem Fall 
allerdings mit nur 3 Freiheitsgraden.
Das ist dann mal ein Test ob das dargestellt wird. Oder auch nicht test








\chapter{Kapitel 3}
\chapter{Anhang}
\begin{thebibliography}{XX}
\bibitem[Bielecki, Rutkowski(2002)]{bielecki02}T. Bielecki, M. Rutkowski, Credit Risk: Modeling, Valuation and Hedging, Springer (2002).
\bibitem[Jarrow, Turnbull(1995)]{jarrow95} R.A. Jarrow, S. Turnbull, Pricing derivatives on financial securities subject to credit risk, {\it Journal of Finance} {\bf 50:1} (1995) 53--85.
\bibitem[Marshall, Olkin(1967)]{marshall67} A.W. Marshall, I. Olkin, A multivariate exponential distribution, {\it Journal of the American Statistical Association} {\bf 62} (1967) pp. 30--44.
\bibitem[Sch\"onbucher(2003)]{schoenbucher03} P.J. Sch\"onbucher, Credit Derivatives Pricing Models, Wiley (2003).\end{thebibliography}


\end{document}